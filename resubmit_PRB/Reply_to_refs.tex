\documentclass[aps,prb,superscriptaddress]{revtex4}
\usepackage{amsmath}
\renewcommand{\topfraction}{0.85}
\renewcommand{\textfraction}{0.1}
\renewcommand{\floatpagefraction}{0.75}
\usepackage[pdftex]{color}
\usepackage{epsfig}
\usepackage{graphicx}% Include figure files
%\usepackage{dcolumn}% Align table columns on decimal point
\usepackage{bm}% bold math
\usepackage{amssymb}%
\usepackage{ulem}
%\normalem
\usepackage[letterpaper,left=2cm,right=2cm,top=2cm,bottom=3cm]{geometry}
\pagestyle{plain}

\newcommand{\bs}[1]{{\boldsymbol{#1}}} 
\newcommand{\red}[1]{{\textcolor{red}{#1}}}
\newcommand{\blue}[1]{{\textcolor{blue}{#1}}}
\newcommand{\magenta}[1]{{\textcolor{magenta}{#1}}}
\newcommand{\green}[1]{{\textcolor[rgb]{0,0.5,0}{#1}}}

\setlength{\parindent}{0pt}

\begin{document}
\begin{center}
{\bf Reply to Referees: Interaction driven phases in the half-filled honeycomb lattice: an infinite density matrix renormalization group study}

\end{center}
\begin{flushright}
Ref: BE13041 / Motruk
\end{flushright}

\vspace{0.2cm}
%\section*{For editors:}


We thank both referees for considering our work sound and relevant.
We address their comments below. Changes in the manuscript file are marked in \blue{blue}.
\\

\section*{Referee $ A$}

\begin{itemize}

\item{\tt In this paper, the authors studied the spinless fermion model on the
honeycomb lattice at half-filling, using iDMRG method. A phase diagram
was also mapped out as functions of V1 and V2. Consistent with
previous studies, the authors claimed the absence of Haldane Chern
insulating phase that was previously proposed by mean-field theory.
Furthermore, two additional symmetry breaking CDW phases are found.
These results are interesting and the paper is well written. However,
I do have a few questions and comments:}

We thank the referee for considering our results interesting and our manuscript well written.

\item{\tt It seems to me that the definition of the charge expectation value
in Eq. (2) is not consistent with what the authors actually calculated
and presented, for instance in Figure 2 and 3. This is because, in
principle, according to the definition in Eq. (2), the charge
expectation value should be larger than zero, i.e., $n_i \geq 0$. It will be
great if the authors could clarify this.}

The charge expectation value is indeed always positive. The numbers $\varrho$ and $\Delta$ in Fig.~2 quantify the deviation from half filling, so $1/2$ has to be added to any density label in Fig.~2. Since $0<\Delta<\varrho<1/2$, the charge expectation value is always positive (see last sentence in the figure caption). The different notation might have been a bit confusing. We added an explanatory sentence in the caption of Fig.~2 to avoid any confusion. In Fig.~3, the area of the blue circles, which is always positive, is proportional to $n$.

\item{\tt In identifying the phase transitions, the authors have utilized the
entanglement entropy and correlation length. However, it is clear that
in the gapless semimetal phase and in the putative continuous phase
transitions, the correlation length is divergent, and the entanglement
entropy is logarithmic divergent (also stated by the authors), which
will be impossible for iDMRG to correctly include in principle, while
a finite bond dimension scaling will be helpful. However, in the
current simulations, the bond dimension is very limited to 1600; the
finite bond dimension scaling was also absent. Some parts of the
conclusion, in particular, the nature of the phase transition lack
clear evidence. Therefore, in order to gain a better insight of the
phase and putative continuous phase transition, the authors should
improve their simulations with larger bond dimension together with a
scaling; for instance, the correlation length vs bond dimension in
Fig.4.}

The computations are very challenging because the only exploitable symmetry is the $U(1)$ charge conservation (in contrast to higher symmetries as for example in $SU(2)$ spin systems) and there are many nearby competing orders in the considered parameter regime. Thus, our simulations are limited to bond dimensions of $\lesssim 2000$.

Concerning the gapless semimetal (SM) phase, it is true that the phase cannot be represented by a state of any finite bond dimension. However, the phase is stable to finite interactions from RG arguments (see Refs.~[35,36]). Since we did not see any signs of phase transitions in the parameter region that we identified as the SM, together with the lack of any order we conclude that the entire region is indeed in the SM phase. The exact point of the phase transition between CDW I and SM on the $V_2=0$ line is indeed very hard to pin down from DMRG results. The transition is known to be of second order from quantum Monte Carlo (QMC) calculations and the correlation length data we present in Fig.~4 is consistent with the QMC results.

To clarify the nature of the semimetal phase, we added an inset in Fig.~4 that shows the finite entanglement scaling for a point in the semimetal phase. We clearly observe the expected scaling of $S \propto \log \chi$ (see Ref.~[42]).



\item{\tt Another problem that limits our understanding of the nature of the
putative continuous phase transition is that the number of data points
is not enough. It will be great if the authors could add more data
points in Fig.~7 and Fig.~8, in particular, when approaching the
putative continuous phase transition.}

We added more data points around the putative continuous transitions in Fig.~7 and 8. The development of correlation length and entanglement entropy when crossing the phase transition is now better visible. 

\item{\tt In section IV, subsection D, the end of the second paragraph, the
authors mentioned that “A first order transition is also consistent
with the entanglement entropy which displays a small jump when
crossing the phase boundary”. However, entanglement entropy results
were not shown. It will be great if the authors could provide these
results.}

We added an inset to Fig.~7 that shows the jump in the entanglement entropy at the Kekul\'e--Semimetal transition.




\end{itemize}

\section*{Referee $B$}

\begin{itemize}

\item{\tt In this paper the authors study the phase diagram of interacting
spinless fermions on the honeycomb lattice by means of the infinite
DMRG method. This model has attracted much attention because of the
possible existence of a Haldane Chern insulator state which was
previously predicted by mean-field theory. However, on the basis of
accurate numerical simulations the authors conclude that this phase is
absent (in agreement also with some of the previous numerical
studies). The authors provide detailed results of the phase diagram of
the model as a function of nearest- and next-nearest neighbor
repulsion strengths. They also reveal two new types of CDW phases
which have not been predicted before.}

\item{\tt The paper is well written and the results are very interesting and
presented in a clear way. Given the high accuracy of their approach
and system sizes which cannot be reached with exact diagonalization,
their results provide a significant contribution in understanding the
physics of this important problem. For these reasons I can recommend
publication of this manuscript in Physical Review B.

I have a few questions and comments which the authors may address in
the revised version:}

We thank the referee for these positive statements and for recommending the publication of our manuscript in Physical Review B. 


\item{\tt Even if the Chern insulator state is not the lowest energy state I
am wondering whether the authors have nevertheless found any signs of
the state, at least as a metastable, competing state? For example,
have the authors tried to initialize the infinite MPS in this phase
and verified that the state does not remain stable upon performing the
optimization?}

We have initialized the algorithm with a Chern insulator state in a parameter region that included the area in which the CI was predicted to occur in previous mean field studies. However, upon performing the DMRG procedure, the state did not prove to be stable and evolved away from the CI phase into the respective phases that we report in our phase diagram. We added a sentence commenting on that to the discussion.
 
\item{\tt The accuracy of the phase boundaries is not very high (due to the
lack of finite-size scaling) for 2nd order phase transitions. This is
however not visible in the phase diagram in Fig.3 where all the
transitions are drawn by sharp lines. It would be good if the authors
could add an estimate of the uncertainty of the phase boundaries by
some error bar (or by making the lines thicker).}

The lines in the phase diagram in Fig.~3 are to be taken with some error. We did not include error bars or thick lines for the sake of better clarity. To clarify this, we now added a sentence in the caption of Fig.~3 indicating that the phase boundaries should be read with an error.

\item{\tt The authors mention on page 4 that “the cylinder geometry
artificially differentiates bonds in its two perpendicular
directions”, and that “this asymmetry vanishes as chi is increased and
thus it is not a physical effect.” I agree that a too small bond
dimension may play a role here, but I assume that even in the exact
limit (chi->infinity) there should be a remaining asymmetry due to the
cylindrical geometry which vanishes only in the infinite Ly limit. Is
this correct?}

We thank the referee for this insight. In the analytically solvable non-interacting case, there is indeed a small asymmetry between bonds around and along the cylinder axis. Since the semimetal phase is adiabatically connected to the non-interacting point, we suppose this is also the case in the entire semimetal region. However, this asymmetry is weaker than the one we observe for low bond dimensions. We have modified our statement in the text and included the fact that has been pointed out by the referee.

\end{itemize}


With the above changes we think we have replied to all comments of the referees and believe that our manuscript is now suited for publication in Physical Review B. 



\end{document}
